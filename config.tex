\newcommand{\red}[1]{\textcolor{red}{#1}}
\newcommand{\orange}[1]{\textcolor{orange}{#1}}
\newcommand{\green}[1]{\textcolor{markgreen}{#1}}
\newcommand{\textgreen}[1]{\textcolor{textgreen}{#1}}
\newcommand{\gray}[1]{\textcolor{lightgray}{#1}}

\newcolumntype{R}{>{\centering\raggedleft\arraybackslash}X}
\newcolumntype{L}{>{\centering\raggedright\arraybackslash}X}
\newcolumntype{C}{>{\centering\arraybackslash}X}

\usepackage[firstinits=true,style=ieee-alphabetic,backend=biber]{biblatex}
\addbibresource{knmopr.bib}
\renewcommand*{\bibfont}{\small}

\usefonttheme{professionalfonts} % using non standard fonts for beamer

\setbeamercovered{invisible}

\newcommand{\textsb}[1]{{\fontfamily{cmss}\fontseries{sbc}\fontshape{n}\selectfont #1}}
\newcommand{\tdots}{\; ..\;}

%ALGORITHMICX
\renewcommand{\algorithmicrequire}{\textcolor{textgreen}{\textbf{Input:}}}
\renewcommand{\algorithmicensure}{\textcolor{textgreen}{\textbf{Output:}}}

% redefine keywords
\algrenewcommand\algorithmicfunction{\textcolor{textgreen}{\textbf{function}}}
\algrenewcommand\algorithmicwhile{\textcolor{textgreen}{\textbf{while}}}
\algrenewcommand\algorithmicfor{\textcolor{textgreen}{\textbf{for}}}
\algrenewcommand\algorithmicif{\textcolor{textgreen}{\textbf{if}}}
\algrenewcommand\algorithmicelse{\textcolor{textgreen}{\textbf{else}}}
\algrenewcommand\algorithmicend{\textcolor{textgreen}{\textbf{end}}}
%\algrenewcommand\algorithmicto{\textcolor{textgreen}{\textbf{to}}}
\algrenewcommand\algorithmicdo{\textcolor{textgreen}{\textbf{do}}}
\algrenewcommand\algorithmicreturn{\textcolor{textgreen}{\textbf{return}}}
\algrenewcommand\algorithmicthen{\textcolor{textgreen}{\textbf{then}}}

% redefine comments
\algrenewcommand{\algorithmiccomment}[1]{{\color{textgreen!50}\textit{//#1}}}

%COLORBOX
\newtcolorbox{mybox}[1]{
colback=chameleongreen2!30,
colbacktitle=chameleongreen1!50,
coltitle=black,
colframe=textgreen,
boxrule=1pt,
titlerule=0pt,
arc=10pt,
title={\strut\textcolor{textgreen}{\textbf{#1}}}
}

\newcounter{definition}
\resetcounteronoverlays{definition}
\newcommand{\define}{\refstepcounter{definition}\thedefinition}
\newenvironment{defi}{
	\begin{mybox}{Definition \define}
}
{\end{mybox}}

\author[Felix Kußmaul]{\Large Felix Kußmaul}
\title[Der KMP-Algorithmus]{\LARGE Der Knuth-Morris-Pratt-Algorithmus}
\subtitle{Einführung und Analyse}
\institute[Uni Siegen]{\normalsize Seminar für Theoretische Informatik\\ Universität Siegen}
\date[29.\ Februar 2016]{\normalsize 29.\ Februar 2016}

\usetheme[pageofpages=/,% String used between the current page and the
                         % total page count.
          titlepagelogo=logo-siegen,% Logo for the first page.
          bullet=triangle,% Use circles instead of squares for bullets.
          titleline=true,% Show a line below the frame title.
          alternativetitlepage=true,% Use the fancy title page.
          ]{Torino}
    
\setbeamercolor{author in head/foot}{bg=chameleongreen3!85}
\setbeamercolor{title in head/foot}{bg=chameleongreen1}
\setbeamercolor{date in head/foot}{bg=chameleongreen4}

\colorlet{markgreen}{chameleongreen3} %chameleongreen3!50!chameleongreen1}
\colorlet{textgreen}{chameleongreen3!70!black}

\setbeamercolor{enumerate item}{fg=textgreen}
%\setbeamercolor{itemize item}{fg=textgreen}

\setbeamertemplate{headline}
{
\hbox{%
  \begin{beamercolorbox}[wd=.28\paperwidth,ht=2.7ex,dp=1.2ex,center]{author in head/foot}%
    \usebeamerfont{author in head/foot}\insertshortauthor\ (\insertshortinstitute)
  \end{beamercolorbox}%
  \begin{beamercolorbox}[wd=.44\paperwidth,ht=2.7ex,dp=1.2ex,center]{author in head/foot}%
    \usebeamerfont{title in head/foot}\insertshorttitle:\ \textbf{\insertsection}
  \end{beamercolorbox}%
  \begin{beamercolorbox}[wd=.28\paperwidth,ht=2.7ex,dp=1.2ex,center]{author in head/foot}%
    \usebeamerfont{date in head/foot}\insertshortdate
  \end{beamercolorbox}}
  \vskip0pt%
}

\DeclareMathAlphabet{\mathcal}{OMS}{cmsy}{m}{n}

\tikzset{
  invisible/.style={opacity=0},
  visible on/.style={alt={#1{}{invisible}}},
  alt/.code args={<#1>#2#3}{%
    \alt<#1>{\pgfkeysalso{#2}}{\pgfkeysalso{#3}} % \pgfkeysalso doesn't change the path
  },
}

\newcommand{\printSectionYes}{\AtBeginSubsection[]
{
 \begin{frame}{Agenda}
 \tableofcontents[sectionstyle=show/shaded,
 					subsectionstyle=show/shaded/hide]
 \end{frame}
}}

\tikzset{onslide/.code args={<#1>#2}{%
  \only<#1>{\pgfkeysalso{#2}}%
}}